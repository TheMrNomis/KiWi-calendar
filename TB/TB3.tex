\documentclass[a4paper,10pt]{article}
\usepackage[utf8]{inputenc}
\usepackage[T1]{fontenc}
\usepackage[french]{babel}
\usepackage{hyperref}

%opening
\title{Tableau de bord \no 3}
\author{S\'ebastien Blin, Merwan Kaf, Amaury Louarn, Paul Perraud}

\begin{document}

\maketitle

\section{Le projet}
\begin{enumerate}
  \item[Titre :] KiwiCalendar (Kiwi Is With Isati)
  \item[Mots-clefs :] Calendrier, évènementiel, semi-collaboratif, Green IT
  \item[Description succinte :] Un calendrier disponible en ligne, regroupant tous les évènements prévus pour les esiriens.
  \item[Contexte du projet :] Dans le cadre du cours d'Innovation et politique, nous devons réaliser un projet d'Innovation. Notre premier projet consistait à contacter le Star et Illenoo afin d'intégrer le wifi dans les bus. Ce projet à été avorté car Illenoo est d\'ej\`a en phase de test afin d'intégrer le wifi dans le bus. Nous avons alors remarqué le manque d'outils pour regrouper les évènements intéressants à la vie de l'école, ce qui nous a amené à choisir ce nouveau projet : Le calendrier semi-collaboratif. En terme de relation avec le Développement Durable, KiwiCalendar permettera de diminuer les mails ainsi que le nombre d'affiches à l'intérieur de l'ESIR.\\
Il s'agit d'une innovation opérationnelle (d'usage) car elle change les habitudes des personnes sur le partage d'évènements.
\end{enumerate}

\section{Description d\'etaill\'ee}
l'ESIR et les Esiriens participent à de nombreux évènements. De plus, la ville de Rennes est très active et propose chaque semaine des évènements dans de nombreux domaines. Pour le moment, ces évènements sont enregistrés dans des calendriers distincts, ou envoyés par mail, ou dans des groupes privés (par exemple, pour les clubs de l'ESIR, il faut soit s'adresser à une personne du club, soit utiliser un outil externe).

L'idée est donc de proposer un calendrier semi-collaboratif disponible en ligne regroupant des évènements intéressants :
\begin{itemize}
  \item De base, les évènements internes à l'école ainsi que ceux du BDE  (soirées, réunions des clubs, etc) seront synchronisés. De plus, de  nombreux évènements rennais seront synchronisés (Diapason, Concerts,  opéras, théatres, cinéma, etc.)
  \item De plus, chaque étudiant pourra ajouter un évènement potentiellement  intéressant pour les étudiants (une conférence intéressante, un  évènement comme le stunfest, etc). (La modération des évènements est un point à travailler)
\end{itemize}
Ce calendrier sera disponible directement à partir du site de l'Isati et  sera exportable dans de nombreux formats. Nous pouvons aussi imaginer un  affichage du calendrier directement sur la télévision du hall de l'ESIR. 
Voici un plan temporaire des catégories
\begin{itemize}
  \item École
  \item Conférences/Évènements (extérieurs)
  \begin{itemize}
    \item Informatique
    \item Biom\'edical
    \item Mat\'eriaux
  \end{itemize}
  \item Clubs et associations (cette catégorie regroupe les associations  internes à l'école ainsi que les associations de Rennes (comme le  labfab, l'association bug, le hackerspace, l'association wikipédia par  exemple)) 
  \item Divertissement / Culture
  \begin{itemize}
    \item Soirée BDE
    \item Diapason
    \item Concerts 
    \item Théatre 
    \item Opéras 
    \item Cinéma
    \item Divers 
  \end{itemize}

\end{itemize}
Des préférences d'affichage peuvent être enregistrées avec notre compte  étudiant.
\section{État de l'art détaillé avec sources}
\subsection*{Agenda du Libre}
Agenda collaboratif regroupant les évènements en rapport avec les logiciels libres en France.\\
\url{http://www.agendadulibre.org/}\\	
\url{https://gitorious.org/agenda-du-libre-rails}
\subsection*{e-agenda}
solution payante d'agenda partagé\\
\url{http://www.e-agenda.fr/}
\subsection*{part-agenda}
agenda partagé payant (solution SaaS)\\
\url{http://www.partagenda.com/}
\subsection*{doodle}
Créer agenda collaboratif -> sondage\\
\url{http://doodle.com/fr/agenda-partage}
\subsection*{multi-planning}
planning gratuit et collaboratif consultable par d'autres personnes selon les droits.\\
\url{http://www.multi-planning.com}
\subsection*{teamcal}
\url{https://github.com/jeanettehuang/teamcal. Pas un calendrier}
\subsection*{Wimi}
Payant, permet de créer un collaboratif avec un système de modération, d'invitation aux évènements. Propose aussi un système de partage de données.\\
\url{https://www.wimi-teamwork.com/fr/gestion-projet/agenda-partage/}
\subsection*{Sogo}
open source\\
\url{http://www.sogo.nu}\\

Tous ces sites proposent des solutions afin de pouvoir mettre plus ou moins facilement en place notre agenda semi-collaboratif. Nous avons choisi de nous baser sur le code de l'agenda du libre, car celui-ci est libre et gratuit. Néamoins ce dernier est codé en Ruby et le site de l'isati ne nous permet pas d'intégrer du Ruby, nous allons donc le codé en php. Mais ces calendriers n'offre pas toujours un choix par calendrier, une authentification par l'ent, pas de modération, etc.


\section{Planning détaillé des séances}
Lors de la séance du 27 mars, nous avons cherché les calendriers facilement exportables, recueilli les avis et les idées (qui se sont avérés positifs) à l'intérieur de la classe. Nous avons commencer à coder l'outil (back-end/front-end) disponible sur : https://github.com/TheMrNomis/KiWi-calendar
\subsection*{10 Avril - Tableau de Bord \no 4}
  \begin{enumerate}
    \item Réalisation de la page web
    \item Mise en place d'un sous-domaine sur le site de l'ISATI et/ou d'une page sur le site de l'ISATI.
  \end{enumerate}

\section{Compte-rendu de la s\'eance}
  \begin{enumerate}
    \item Avancé sur la réalisation de la page web.
    \item Recherche sur comment récupérer les événements de la ville de Rennes.
    \item Présentation du projet aux autres groupes.
  \end{enumerate}

\section{Cahier des charges}
Voir fichier cahier des charges

\end{document}
